\documentclass[11pt,a4paper]{moderncv}
\moderncvtheme[orange]{classic}                
\usepackage[utf8]{inputenc}
\usepackage[top=1.0cm, bottom=1.0cm, left=2cm, right=2cm]{geometry}
\usepackage{./moderntimeline}

\setlength{\hintscolumnwidth}{2.7cm}

\firstname{Pierre-Alexandre}
\familyname{VEYRY}
\title{Étudiant en ingénierie informatique}             
\address{Les jardins de Paste}{07000 PRIVAS}
\mobile{06.48.79.87.79}                    
\email{veyry\_p@epita.fr}
\extrainfo{20 ans -- Permis B}           

\begin{document}

\tlmaxdates{2012}{2015}
\tlwidth{0.8ex}
\tltext{\tiny}

\maketitle

\section{Cursus}


\tlcventry[orange]{2014}{2014}{Systems Engineer}{\href{http://www.orness.com}{ORNESS} then \href{http://www.alten.fr}{Alten}}{Sophia Antipolis}{}
{Consultant at France T\'el\'ecom
\begin{itemize}
 \item Administration of Cfengine, FAI and the software package repository;
 \item Renewal and industrialization of the software deployment system for Debian/Ubuntu;
 \item Management of the Gforge (collaborative development web interface) platform;
 \item Monitoring of the fleet of servers;
 \item Writing of technical documentation.
\end{itemize}}


\cventry{Janvier 2014 \\à Mai 2014}{Echange international}{Stafforshire University}{}{}{Stafford, Angleterre}

\cventry{2012 - 2017}{Diplôme d'ingénieur en informatique}{EPITA}{Le Kremlin-Bicêtre (94)}{}{Ecole pour l'Informatique et les Techniques Avancées}

\cventry{2009 - 2012}{Baccalauréat}{Mention Très Bien}{Privas (07)}{}{Série scientifique, spécialité Physique-Chimie}


\section{Expériences}
	\subsection{Projets scolaires}
	\cventry{2014}{Projets scolaires de première année de cycle ingénieur}{}{}{}{Fnmatch, TinyPrintf, Minimake, MyReadIso, Malloc, Formula One, Raytracer, myHTTPd}
	\cventry{2013}{OC'air}{Logiciel d'OCR (Optical Character Recognition) développé en OCaml}{}{}{\url{http://ocair.areswar.eu}}
	\cventry{2012 - 2013}{Metastruggle}{Jeu vidéo en 3D réalisé en C\# avec le framework XNA}{}{}{\url{http://fr.metastruggle.eu}}

	\subsection{Expériences professionnelles}
	
\cventry{Septembre 2014 \\à Aujourd'hui}{Enseignant assistant (ASM)}{EPITA}{Villejuif}{France}{
Encadrement des travaux pratiques (programmation en langage C) des élèves de deuxième année de l'école EPITA}
\cventry{Mai 2014 \\à Juillet 2014}{Développeur stagiaire}{SYDER}{69570 Dardilly}{}{
\begin{itemize}%
\item Développement d'un service Web de consultation de données et de simulation avec PHP et Javascript,
\item Refonte du design du portail Web de l'entreprise et conception de logos thématiques pour chaque service Web de l'entreprise,
\item Réalisation d'un système très simple de cartographie à partir d'Openlayers et OpenStreetMap pour un des services Web de l'entreprise
\item Mise en place de machines virtuelles sous GNU/Linux (Debian Squeeze) en vue de l'installation de nouveaux services
\end{itemize}}

\section{Compétences techniques}
	\cvitem{Langages}{C, OCaml, C\#, PHP, Javascript, Python, SQL,  HTML, CSS, \LaTeX}{}{}{}{}
	\cvitem{Systèmes d'exploitation}{GNU/Linux, BSD, Mac OS X, Windows}{}{}{}{}
	\cvitem{Bases de données}{MySQL, Microsoft SQL Server}{}{}{}{}


\section{Langues}
\cvlanguage{Français}{Langue natale}{}
\cvlanguage{Anglais}{TOEIC Niveau 945}{}
\cvlanguage{Allemand}{Niveau B2}{}

\section{Centres d'intérêt}

\cvitem{Arts}{Piano (Cycle III), Musique, Cinéma, Littérature, Histoire de l'Art}{}{}{}{}

\cvitem{Sports}{Triathlon, Canoë-Kayak}{}{}{}{}


\end{document}
