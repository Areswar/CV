\documentclass[11pt,a4paper]{moderncv}
\moderncvtheme[red]{classic}
\usepackage[utf8]{inputenc}
\usepackage[top=0.5cm, bottom=0.5cm, left=2cm, right=2cm]{geometry}
\usepackage{./moderntimeline}

\setlength{\hintscolumnwidth}{2.7cm}

\firstname{Pierre-Alexandre}
\familyname{VEYRY}
\title{Étudiant en ingénierie informatique}
\address{Apt 308, 14 rue Carnot}{94800 VILLEJUIF}
\mobile{06.48.79.87.79}
\email{veyry\_p@epita.fr}
\homepage{http://pa.veyry.fr}
\extrainfo{21 ans -- Permis B}

\begin{document}

\tlmaxdates{2011}{2016}
\tlwidth{0.8ex}
\tltext{\tiny}

\maketitle

\vspace*{-0.5cm}

\section{Cursus}
    \tlcventry{2012}{0}{École d'ingénieurs en Informatique}{EPITA}{Le Kremlin-Bicêtre}{}{}
    \tldatecventry{2014}{Échange international d'un semestre}{Staffordshire University}{Royaume-Uni}{}{}
    \tldatecventry{2012}{Baccalauréat Scientifique}{Mention Très-Bien}{}{}{}


\section{Expériences}
    \subsection{Emplois et stages}
        \tlcventry{2015}{2016}{Assistant enseignant (YAKA)}{EPITA}{Villejuif}{France}
            {Enseignement de C++, JAVA, et SQL aux étudiants de troisième année à l'EPITA}
        \tldatecventry{2015}{Stage R\&D de 4 mois}{Aerys}{Paris}{France}{Recherche et développement sur le moteur 3D Minko (C++)}
        \tlcventry{2014}{2015}{Assistant enseignant (ASM)}{EPITA}{Villejuif}{France}
            {Enseignement du C et d'UNIX aux étudiants de deuxième année à l'EPITA}
        \tldatecventry{2014}{Développeur Web Stagiaire}{SYDER}{Développement de différentes applications Web en HTML, CSS, Javascript et PHP}{}{}

    \subsection{Projets}
        \tldatecventry{2015}{Viola-Jones}{Programme de reconnaissance faciale parallélisé en C++}{}{}{}
        \tldatecventry{2015}{Tiger Compiler}{Compilateur pour le langage Tiger en C++}{}{}{}
        \tldatecventry{2014}{42sh}{Implémentation d'un shell UNIX (\texttt{bash ----posix}) en C en groupes de quatre}{}{}{}
        \tldatecventry{2014}{HTTPd}{Serveur HTTP en C}{}{}{}
        \tldatecventry{2014}{Malloc}{Implémentation de Malloc selon le modèle Best-Fit en C}{}{}{}
        \tldatecventry{2013}{OCR}{Logiciel de reconnaissance de caractères en OCaml}{}{}{}
        \tlcventry{2012}{2013}{Jeu vidéo}{Metastruggle}{Jeu vidéo de combat en 3D développé en C\# avec XNA}{}{}


\section{Compétences techniques}
    \subsection{Langages de programmation}
        \cvitem{Compétent}{C++, C, JAVA, PHP, JS, SQL, \LaTeX}{}{}{}{}
        \cvitem{Expérimenté}{C\#, OCaml, HTML, CSS}{}{}{}{}
        \cvitem{Débutant}{Python, Swift, Go}{}{}
    \subsection{Divers}
        \cvitem{Systèmes}{GNU/Linux, Mac OS X, Windows, BSD}{}{}{}{}
        \cvitem{Logiciels}{Git, Blender 3D, Adobe Photoshop, Gimp, Steinberg Cubase}{}{}{}{}


\section{Langues}
    \cvlanguage{Français}{Langue natale}{}
    \cvlanguage{Anglais}{Avancé (TOEIC:980/990)}{}
    \cvlanguage{Allemand}{Notions}{}


\section{Centres d'intérêt}
    \cvitem{Arts}{Piano (Pratiqué pendant 11 ans), Musique, Cinéma, Littérature, Histoire de l'Art}{}{}{}{}
    \cvitem{Sports}{Triathlon, Canoë-Kayak}{}{}{}{}


\end{document}
